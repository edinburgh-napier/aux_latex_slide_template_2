\documentclass{beamer} %Uncomment for presentation
%\documentclass[handout]{beamer} %Uncomment for printable handouts
%\documentclass[handout, notes=show]{beamer} %Uncomment for speaker notes
\usepackage{handoutWithNotes}
% Uncomment for handout with note space
%\pgfpagesuselayout{3 on 1 with notes}[a4paper, border shrink=5mm] 
\usetheme{Pittsburgh}
\usefonttheme{professionalfonts}
\usecolortheme{beaver}
\usecolortheme{rose}
\setbeamertemplate{items}[circle]
\setbeamercolor{item projected}{bg=darkred}
\setbeamercolor{itemize item}{fg=darkred}
\setbeamercolor{itemize subitem}{fg=darkred}
\setbeamercolor{itemize subsubitem}{fg=darkred}
\setbeamercolor{block title}{bg=gray}
\setbeamercolor{block title}{fg=black}

\author[Shortened Name]{[Lecturer Name]}
\title[Shortened Title]{[Title of Presentation]}
\subtitle[Shortened Subtitle]{[Subtitle of Presentation]}
\institute[Edinburgh Napier]
{
    School of Computing\\
    Edinburgh Napier University\\
    Edinburgh\\[1ex]
    \texttt{[Email Address]}\\
    \includegraphics[width=4cm]{logo}
}
\date{}

\AtBeginSection[]
{
    \begin{frame}{Outline}
        \tableofcontents[currentsection]
    \end{frame}
}

\usepackage{hyperref}
\usepackage{graphics}
\usepackage{color}

\definecolor{grey}{rgb}{0.9, 0.9, 0.9}
\definecolor{dkgreen}{rgb}{0,0.6,0}
\definecolor{dkred}{rgb}{0.6,0,0.0}

\begin{document}

    \begin{frame}[plain]
        % Creates the title frame.
        \titlepage
    \end{frame}
    
    \begin{frame}{Outline}
        % Creates table of contents.
        % Remove [pausesections] to stop a slide being generated for each item.
        \tableofcontents[pausesections]
    \end{frame}
    
    \begin{frame}{Frame}
        A single slide is defined as a frame.
        
        The title of the frame is in the second set of curly braces.
    \end{frame}
    
    \begin{frame}{Bullet Points}
        \begin{itemize}
            \item Bullet points are added via the itemize environment.
            \item We can also have sub-bullets by adding a further itemize within the first itemize.
            \begin{itemize}
                \item Like so.
                \item This can be continued, although the template only supports a few levels simply.
            \end{itemize}
        \end{itemize}
        \begin{enumerate}
            \item Numbered lists are added using the enumerate environment.
            \begin{enumerate}
                \item And we can go down levels as well.
            \end{enumerate}
            \begin{itemize}
                \item Or go into bullet points.
            \end{itemize}
        \end{enumerate}
    \end{frame}
    
    \begin{frame}{Revealing}
        \begin{itemize}
            \item It is possible to reveal bullet points one ``slide'' at a time.
            \item For example:
            \item<2-> This appears as the second slide.
            \item<4-> This the fourth.
            \item<3> And this as third only.
        \end{itemize}
    \end{frame}
    
    \begin{frame}{Auto-reveal}
        \begin{itemize}[<+->]
            \item It is possible to automatically have the bullet points revealed one at a time.
            \item To do this, we specify a modifier on the itemize environment - \[<+->\]
            \item Then everything is automatically generated.
        \end{itemize}
    \end{frame}
    
    \section{A Section}
    
    \begin{frame}{Sections and the Outline}
        \begin{itemize}
            \item To add something to the Outline slide (table of contents) create a section.
            \item Each section will add an entry to the Outline.
            \item Usually best to generate the slides twice for this.
        \end{itemize}
    \end{frame}
    
    \begin{frame}{Images}
        \begin{itemize}
            \item Adding images is just the same as normal LaTeX.
            \item Just add a graphics include where you want the image, like so:
        \end{itemize}
        \includegraphics[width=\textwidth]{logo}
    \end{frame}
    
    \begin{frame}{Columns}
        \begin{columns}
            \begin{column}{.5\textwidth}
                \begin{itemize}
                    \item Columns can be added using the columns environment.
                    \item This allows some layout control of your slides.
                    \item For example, you might want a picture on the right and text on the left as we have here.
                    \item Each column has to be specified using the column environment.
                \end{itemize}
            \end{column}
            \begin{column}{.5\textwidth}
                \includegraphics[width=\textwidth]{logo}
            \end{column}
        \end{columns}
    \end{frame}
    
    \begin{frame}{Blocks}
        \begin{itemize}
            \item You might also want to have a text box (or similar) to highlight something on the slide.
            \item For this, use the block environment
        \end{itemize}
        \begin{block}{Block Title}
            Blocks can also be revealed using the same technique as used for itemize, etc.
        \end{block}
    \end{frame}

\end{document}
